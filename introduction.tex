\chapter{Introduction}

According to one of the report of US, around one billion people travel on airlines each year.

It is important to realize the on board medical resources are limited.

Environmental changes such as when moving from sea level to high altitude, the level of oxygen gradually decreases as the air thins out, including the reduction of atmospheric pressure, temperature and humidity. As a result, the heart rate increases as it tries to deliver more oxygen to the muscles.

Increases in high altitude increases the frequency of  

Flying at high altitude can create serious heart problems. As the altitude increases, the level of oxygen in blood falls down which can lead to fainting or even heart attacks among some passenger.



An arrhythmia can be harmless or a life threating Therefore, for pilot's a thorough medical evaluation is necessary to assess the severity of an arrhythmia for the safety of both pilot and the passengers.





Health care is one of the hottest research areas in the current era. Monitoring of vital signs parameter like respiration, ECG, EEG, temperature, heart rate, etc,. getting more and more important in every field of life. The most focus is towards the wearable and wireless sensor as they do the minimal obstruction in day to day life.

A variety of technologies are already in used to measure the vital signs. They include traditional stethoscopes, electrodes for measuring ECG and different types of gauges, but they have their drawbacks in terms on comfort and convenience. For example, to measure the ECG, the electrodes are required to be directly attached to the skin of the patient, which is very inconvenient and limits the patient's movement. Strain gauges uses the belts to measure the blood pressure and respiration which again limits the daily activities. Even though these technologies are reliable and provide better results, but they inadequate for long term everyday activities.

Contact-less sensors are the next big thing in the health care. They are designed in such a way that they can disappear in the daily surroundings without any disruption. Different techniques has been used to integrate contact-less sensors into clothes, chairs, smart watches, smart phones, shoes, beds, etc ot measure different vital signs.


As mentioned earlier, most of the work done in the health care technologies have uses the sensors which are required to be placed on the body, therefore, a trends need to be shift towards contact-less sensors as they are much more convenient to use. Moreover, not much work has been carried out to use the deep learning technologies along with contact-less sensor to identify the cardiac arrhythmias in real-time environment. 

The aim of this thesis, is to build a system that provides a health care solution which would be able to track the pilot's vital signs including heart rate, temperature, ECG, etc, as well as can detect the cardiac arrhythmias in real-time. The early identification of disease can help to provide quick treatment to the pilot.

Multiple contact-less sensors are used in the development of the system to measure the various vital signs of pilots. These sensors include:

\begin{enumerate}
	\item Capactive ECG sensor
	\item Photoplethysmogram sensor
	\item Magnetic impedance sensor
	\item Ballistocardiogram sensor
	\item Thermal camera
\end{enumerate}

These sensors will be installed in the pilot seat so they can continuously track the health status of pilot during a course of flight, and in case of any irregularity, immediate actions can be taken for the safety of the pilot and of course for the passengers as well.

A deep learning model has been trained using convolutional neural network.  Various cardiac arrhythmias dataset were taken from already existed dataset to train the model. The advantage of using CNN is that unlike other machine learning algorithms, it does not require a feature extraction phase.

Multiple platforms have been used along with tablets to visualize the results and vital signs of the passengers. A cloud has been setup to store all the sensors data and vital signs so, the data can be accessed from anywhere in the world.
The same data later can also be used for different purpose such as, for the training of deep learning model.



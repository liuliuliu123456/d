% !TEX root = ../main.tex

%\chapter{Einleitung}
\chapter{Introduction}
\label{sect:introduction}

% Thema motivieren, worum geht es und warum ist das wichtig?
% Motivate the topic, what is this thesis about and why is that important?

An introduction motivates the topic of the thesis and explains the origin of the task. \todo{Motivate the topic, what is this thesis about and why is that important?}
It starts often with a very general statement followed by explanations which difficulties (of course with relevance to the thesis topic) arise.
Finally, the difficulties are narrowed down to the topic of the thesis.
At this point the reader should understand the relevance of the problem being addressed by the thesis.
An example for an introduction could look like the text following below. \\

At the end of their bachelor or master studies young students have to write a final thesis.
For many of them this is their first written academic document of mentionable relevance.
Thus, those students have no experience with academic writing and therefore require often some basic guidelines.
Guidelines are usually provided directly by the thesis advisers up front and/or during thesis review.
Providing such guidelines in a written and structured form to the students results hopefully in less review work for advisor and student, more consistent and clear theses and hence theses of better quality. \todo{Dies ist eine Anmerkung was noch zu machen ist.}

%\section{Aufgabenstellung}
section{Main contributions}
\label{sect:main_contributions}
% Was soll gemacht werden?
% What shall be done within this thesis
This document aims at providing a basic template for students writing their bachelor or master thesis at \abk{i11}.
It shall provide a basic overview on how such writings are structured and provide some useful hints.

%\section{Gliederung}
\section{Outline}

% Wie ist die Arbeit aufgebaut? Dies sollte den roten Faden beschreiben der sich durch die ganze Arbeit zieht und den Zusammenhang der einzelnen Teile verdeutlichen.
% How is this document structured? 

The outline of a thesis should point out how the different chapters are related to each other and that the thesis is well structured.
It should show that the different chapters form a single coherent document.
An example how such an outline could look like for this document is given below.\\

Chapter~\ref{sect:basics} gives an introduction and overview on latex citations and figures to provide some basic knowledge how to handle those latex-constructs.
The proximate Chap.~\ref{sect:relatedWork} presents an overview of work related to this document.
Chapter~\ref{sect:corechapter} introduces subsequently how citations have to be used in order to avoid wrong citation styles or even plagiarism.
It therefore requires information introduced before in chapter \ref{sect:basics}.
Additionally, this chapter provides some hints dealing with figures in latex and their quality.
The document is concluded by Chap.~\ref{sect:conclusion} which gives also an outlook how this document could be extended in the future.

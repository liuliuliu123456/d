% !TEX root = ../main.tex

\chapter{Appendix1}
\label{app:a}

Here could be a large figure or a very big table like this:

\begin{tabularx}{\textwidth}{| p{0.2\textwidth} | p{0.4\textwidth} | p{0.2\textwidth} |}%{\linewidth}{|X|X|}
  \caption{Table caption} \\\endfirsthead
	
  \hline
  \textbf{column1} & column2 & column3 \\ \hline
  row 1 & X & X \\ \hline
	row 2 & X & X \\ \hline
	row 3 & X & X \\ \hline
	row 4 & X & X \\ \hline
	row 5 & X & X \\ \hline
	row 6 & X & X \\ \hline
	row 7 & X & X \\ \hline
	row 8 & X & X \\ \hline
	row 9 & X & X \\ \hline
	row 10 & X & X \\ \hline
	row 11 & X & X \\ \hline
	row 12& X & X \\ \hline
	row 13 & X & X \\ \hline
	row 14 & X & X \\ \hline
	row 15 & X & X \\ \hline
	row 16 & X & X \\ \hline
	row 17 & X & X \\ \hline
	row 18 & X & X \\ \hline
	row 19 & X & X \\ \hline
	row 20 & X & X \\ \hline
	row 21 & X & X \\ \hline
	row 22 & X & X \\ \hline
	row 23 & X & X \\ \hline
	row 24 & X & X \\ \hline
	row 25 & X & X \\ \hline
	row 26 & X & X \\ \hline
	row 27 & X & X \\ \hline
	row 28 & X & X \\ \hline
	row 29 & X & X \\ \hline
	row 30 & X & X \\ \hline
	row 31 & X & X \\ \hline
	row 32 & X & X \\ \hline
	row 33 & X & X \\ \hline
 \end{tabularx}


% ----------------------------------------------------------------------------
\chapter{Digital Mediums as part of the thesis}
\label{app:b}

This appendix is not empty but not referenced before.
Such things shall not happen in a final version of a thesis.
The same applies for figures, tables and other provided materials.

If you provide digital data with the printed version of you thesis (e.g. a CD/DVD) the contents of the digital recording have to be organized in a structured way to.
The digital medium should contain a README file in the root folder.
This textfile (*.txt) should state how the data is organized on the medium.
Potential content for an attached digital recording is a digital version of the thesis or digital copies of the web-sources (can be created using a pdf printer).
In case the digital medium contains program code or executables form third parties, please ensure that you do not violate any licenses.

% -----------------------------------------------------------------------------
\chapter{Use of Symbols, Abbreviations and Index}
\label{app:c}

Symbols can be introduced using the following Latex commands:
\begin{flushleft}
\textbackslash newglossaryentry\{symbol:pi\}\{
	name=\textbackslash ensuremath\{\textbackslash pi\},\\
	description=\{Kreiszahl [einheitenlos]\},\\
	sort=symbolpi,type=symbolslist,\\
\}
\end{flushleft}

The created symbol can then be referenced by:
\begin{flushleft}
\textbackslash sym\{pi\}
\end{flushleft}
which will be displayed as \sym{pi}.
Symbols being created this way are automatically added to the list of symbols behind the list of contents.


In a similar manner abbreviations can be used.
To define a new abbreviation, use
\begin{flushleft}
\textbackslash newacronym\{i11\}\{I11\}\{Lehrstuhl Informatik 11\}
\end{flushleft}
which defines a new acronym.
It can be used with the command \textbackslash abk\{i11\} and looks like \abk{i11} when referenced multiple times.
The first use of the command, however, will appear as can be seen in the first sentence of section \ref{sect:main_contributions}.
For the full command-reference see the documentation of the acronym package.
Alternatively, see \url{http://texblog.org/2014/01/15/glossary-and-list-of-acronyms-with-latex/} for explanations to capitalize and pluralize acronyms.


An index (germ. Stichwortverzeichnis) can be created using the \textbackslash printindex command.
To add text phrases to the index write \textbackslash index\{phrase for the index\} which will be displayed as \index{phrase for the index} but create an entry in the index chapter referring to the page the phrase is placed to (see next page).
The index creation requires two latex compilations and the use of makeindex.
However, this should be no issue using this template and compiling with compile.bat.
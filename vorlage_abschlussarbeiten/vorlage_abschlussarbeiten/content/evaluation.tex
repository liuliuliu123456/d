% !TEX root = ../main.tex

\chapter{Evaluation}
\label{sect:evaluation}

Since there is no evaluation for this document, this chapter lists some points which may be considered when evaluating software/algorithms/procedures.
\begin{itemize}
	\item Comparison with other approaches to achieve the same goal as this work
	\item Measure resources needed by a developed tool/procedure/etc. regarding (time consumption, cpu usage, memory usage, ...)
	\item What about the quality of the results?
	\item Are there any problems with the procedure/applicability or restrictions?
	\item Evaluate (if possible) using \enquote{real world examples} and not only academic ones.
\end{itemize}

Please note that they do not all have to apply or be suitable for any thesis topic.
It might be that evaluation results are not presentable in a structured and readable way.
For instance, if tables become to large.
In such cases it might be helpful to place the results in the appendix \ref{app:a} and refer to them.
This applies also for figures, test specifications and tables at any other place in the thesis and is the only exception allowing forward references.
The evaluation should state very clear what is possible and what is not.
If there are any, the limits of the developed approach should be shown using suitable examples.
Do not be afraid to show limitations of approaches.
This does generally not undermine the authors achievement if it is clear and can be explained why the limitations exist.
The author should avoid to use vague words like better, worse, etc. to describe the evaluation results.
This applies in general for the other chapters of the thesis, too.

The next chapter of this document is the final chapter, which should always conclude the work presented before.